% ------------------------------------------------------------------------ %
% !TEX encoding = UTF-8
% !TEX TS-program = pdflatex
% !TEX root = ../Project.tex
% !TEX spellcheck = en-EN
% ------------------------------------------------------------------------ %
%
% ------------------------------------------------------------------------ %
% 	CHAPTER TITLE
% ------------------------------------------------------------------------ %
%
\chapter{Design and execution\\of the study}
%
%
% ------------------------------------------------------------------------ %
%
\section{Evaluation method adopted}
For the evaluation of the usability of ``The Big Family" site is used the user testing method. With this approach usability properties are retrieved analyzing the interaction between some representatives of real users and the system. The test is performed as the developers of the site want to gather data in order to improve their product.
%
% ------------------------------------------------------------------------ %
%
\section{Task scenarios defined for the test}
\begin{itemize}
\item Task 1: You stumbled upon the website of ``The Big Family", understand what is the scope of this association.
\item Task 2: You are helping a friend looking for services for disabled preople near Certaldo. Identify the ones offered by the association.
\item Task 3: You want to write an email to have some more information about the association. Find the address.
\item Task 4: You need to contact someone working in the Pet Therapy service. Find a telephone number to call.
\item Task 5: You want to visit the association's site nearest to your house. Find where it is.
\end{itemize}
%
% ------------------------------------------------------------------------ %
%
\section{Partecipants for the test}
For this test the partecipants are 5 users and 1 moderators.\\
The users are recruited in order to best match the possible target audience for the system, and this was done considering that the site is about an association that holds some children care center on the territory.\\
The user profiles found in the audience of the association are: 
\begin{itemize}
\item Young adult
\item Adult parent of a child
\item Retired
\end{itemize}
The user goal is to:
\begin{itemize}
\item Find information about the association, since one of their relatives is a children with disability and the association could be useful for him/her
\end{itemize}
The moderator is one the developers of the site, alternately, depending on the disponibility of them.
%
% ------------------------------------------------------------------------ %
%
\section{Usability variables to be measured}
For each user and task, the moderators gather quantitative and qualitative indicators.\\
Quantitative indicators:
\begin{itemize}
\item Efficiency (time used for every task)
\item Effectiveness (task completion (with or without assistance))
\item Number of errors
\item Task success rate
\end{itemize}
Qualitative data:
\begin{itemize}
\item What is liked/disliked
\item Disorientation (information not found)
\item Frustration
\end{itemize}
%
% ------------------------------------------------------------------------ %
%
\section{How the test was performed}
For the execution of the test a typical context of use of the site is simulated (i.e on the dining room table or in the living room) and every user is requested to work with a laptop to perform the tasks predefined by the moderator.\\
Before starting the test, its steps and its purpose are explained, the user is set at his ease and is told that he can leave whenever he wants.
The sheet with the tasks is given to the user that can start to read and to ask what he doesn't understand of it or whatever comes to his mind. At this point the actual test can start. Since every profile of the audience can need and perform any action on the site, all predefined tasks are requested to every user. As long as the user performs the various tasks the moderator is present to gather the data needed and to observe any obstacle for the users to reach their goals.\\
After the test a simple questionnaire is delivered to the user to understand his opinion about the usability of the system.

%
%
% --------------------------------END------------------------------------- %
