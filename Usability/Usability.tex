% ------------------------------------------------------------------------ %
% !TEX encoding = UTF-8
% !TEX TS-program = pdflatex
% !TEX root = Usability.tex
% !TEX spellcheck = en-EN
% ------------------------------------------------------------------------ %
%
%
% ------------------------------------------------------------------------ %
% 	PREAMBLE
% ------------------------------------------------------------------------ %
%
\documentclass
	[12pt,	
	a4paper,		%
	twoside,		% fronte-retro
	openany,
	titlepage,% 	% nuova pagina dopo il titolo (necessario per frontespizio)
	]{book}
%
% ------------------------------------------------------------------------ %
%
\usepackage[T1]{fontenc}		% codifica di output
%
\usepackage[utf8]{inputenc}		% codifica di input; anche [latin1] va bene
%
\usepackage[italian,english]{babel}	% languages
%
\usepackage{csquotes}
%
\usepackage{microtype}		% micro-tipografia
%
\usepackage{lmodern}
%
\usepackage{pdflscape}
%
% ------------------------------------------------------------------------ %
%
% 	LAYOUT - MARGINS - BINDING
%
% -- MANUAL (PoliMi settings)
\usepackage{geometry}
\geometry{verbose,	% verbose = displays the parameter results on the terminal
	top=43mm,		% upper margin (PoliMi=43mm)
	bottom=44mm,	% bottom margin (PoliMi=44mm)
	inner=37mm,		% inner margin (PoliMi=41mm)
	outer=37mm,		% outer margin (PoliMi=32mm)
%	bindingoffset=5mm,	% binding margin
	heightrounded}
%
% ------------------------------------------------------------------------ %
%
\usepackage{multicol}
%
\usepackage{changepage,calc}                 % centra il frontespizio
%
\usepackage{emptypage}		% pagine vuote senza testatina e piede di pagina
%
\usepackage{indentfirst}	% rientra il primo paragrafo di ogni sezione
%
\usepackage{booktabs}		% tabelle (\toprule, \midrule, \bottomrule)
%
\usepackage{tabularx}		% tabelle di larghezza prefissata
%
\usepackage{graphicx}		% immagini
%
\usepackage[figuresright]{rotating}	% tabelle a 90 gradi
%
\usepackage{subfig}			% sottofigure, sottotabelle
%
\usepackage{caption}		% didascalie
%
\usepackage{listings}		% codici

\lstset{ %
  language=Java,                  % the language of the code
  basicstyle=\footnotesize,       % the size of the fonts that are used for the code
  numbers=left,                   % where to put the line-numbers
  numberstyle=\tiny\color{gray},  % the style that is used for the line-numbers
  stepnumber=1,                   % the step between two line-numbers. If it's 1, each line 
                                  % will be numbered
  numbersep=5pt,                  % how far the line-numbers are from the code
  backgroundcolor=\color{white},  % choose the background color. You must add \usepackage{color}
  showspaces=false,               % show spaces adding particular underscores
  showstringspaces=false,         % underline spaces within strings
  showtabs=false,                 % show tabs within strings adding particular underscores
  frame=single,                   % adds a frame around the code
  rulecolor=\color{black},        % if not set, the frame-color may be changed on line-breaks within not-black text (e.g. commens (green here))
  tabsize=4,                      % sets default tabsize to 2 spaces
  captionpos=b,                   % sets the caption-position to bottom
  breaklines=true,                % sets automatic line breaking
  breakatwhitespace=false,        % sets if automatic breaks should only happen at whitespace
                   % show the filename of files included with \lstinputlisting;
                                  % also try caption instead of title
  keywordstyle=\color{blue},          % keyword style
  commentstyle=\color{gray},       % comment style
  stringstyle=\color{mauve},         % string literal style
  escapeinside={\%*}{*)},            % if you want to add a comment within your code
  morekeywords={*,...}               % if you want to add more keywords to the set
}
%
\usepackage[font=small]{quoting}	% citazioni
%
\usepackage{amsmath,amssymb}	% matematica
%
\usepackage{mathtools}		% matematica
%
\usepackage{amsthm}			% matematica
%
\usepackage[output-decimal-marker={,}]{siunitx}	% SI (con separatore decimale=virgola)
%
\usepackage[english]{varioref}		% riferimenti completi, con indicazione della pagina (\vref)
%
\usepackage{mparhack}	% finezze tipografiche (bug fixes di LaTeX)
%
\usepackage{relsize}			% make text larger or smaller than the surrounding text
% 				% \larger[i] \smaller[i]
%
% ------------------------------------------------------------------------ %
%
% 	BIBLIOGRAPHY
%
%
% biblatex package
%
% STILI di citazione:
% style=numeric-comp,	<-- ufficialmente richiesto dal PoliMi (numeri tra [ ])
% style=philosophy-modern,	<-- autore-anno (meno anonimo, pi� immediato e pi� elegante)
%
\usepackage[style=philosophy-modern,	% numeric-comp oppure philosophy-modern,
	hyperref,			% clickable references
	backref,			% link alle pagine in cui il riferimento � citato
	natbib, 			% mantiene compatibilit� con eventuali comandi natbib
	backend=biber,		% motore bibliografico (v. ArteLatex di Pantieri)
	defernumbers=true,	 	% riferimenti ordinati in ordine di comparsa
	]{biblatex}
%
\addbibresource{Bibliografia.bib}	% database bibliografico
%
% ------------------------------------------------------------------------ %
%
% Per generare effettivamente la bibliografia nel documento
% questa e` la sequenza di composizione:
% 1. si compone il documento con LATEX una prima volta;
% 2. si lancia il programma Biber premendo l�apposito pulsante dell�editor;
% 3. si compone il documento altre 2 volte con LATEX (ma anche 3, NdA)
% Tale sequenza deve essere ripetuta solo se vengono fatte modifiche/aggiunte
% al database bibliografico.
%
% ------------------------------------------------------------------------ %
%
\usepackage[dvipsnames]{xcolor}	% colori - 68 colori predefiniti:
% 								% http://en.wikibooks.org/wiki/LaTeX/Colors
%
\usepackage{lipsum}			% testo fittizio
%
\usepackage{eurosym}		% simbolo dell'euro
%
\usepackage{hyperref}		% collegamenti ipertestuali
\hypersetup{
    colorlinks = false,
    linkbordercolor = {white}
}
%
\usepackage{bookmark}		% gestione segnalibri del PDF
%
\usepackage{guit}			% simboli del Guit
%
\usepackage{fancyhdr}		% testatine e piede personalizzati
\pagestyle{fancy}
\fancyhead[LO,LE]{\slshape \leftmark}
\fancyhead[RE,RO]{\slshape \rightmark}
\fancyfoot[LO,LE]{\slshape Authors: Alessandro Aimi, Roberto Bigazzi}
\fancyfoot[RO,RE]{\slshape \thepage}
\cfoot{}
\renewcommand{\chaptermark}[1]{\markboth{\MakeUppercase{\thechapter.\ #1}}{}}
\renewcommand{\sectionmark}[1]{\markright{\thesection.\ #1}}
\renewcommand{\footrulewidth}{0.4pt}% default is 0pt
\setlength{\headheight}{15pt}

\fancypagestyle{plain}{
\fancyfoot[LO,LE]{\slshape Authors: Alessandro Aimi, Roberto Bigazzi}
\fancyfoot[RO,RE]{\slshape \thepage}
\cfoot{}
\renewcommand{\footrulewidth}{0.4pt}% default is 0pt
}
%
\usepackage{colortbl}		% per colorare i filetti delle tabelle
%
\usepackage[footnote,		% acronym description in the footer
			smaller,		% smaller acronyms size
			]{acronym}		% acronyms
%
\usepackage{multirow}		% celle tabelle alte pi� di una riga
%
\usepackage{pdfpages}		% adding external pdf files
%
%
\renewcommand{\thesection}{\thechapter.\Alph{section}}
%
% ------------------------------------------------------------------------ %
% 	BEGIN DOCUMENT
% ------------------------------------------------------------------------ %
%
\begin{document}
%
% ------------------------------------------------------------------------ %
% 	FRONTMATTER
% ------------------------------------------------------------------------ %
%
\frontmatter
%
% Frontispiece
\includepdf[pages={1}]{FrontMatter/Frontispiece.pdf}
%
\let\cleardoublepage\clearpage
%
\clearpage
\setcounter{page}{1}
\pagenumbering{Roman}
\tableofcontents
%
%\input{FrontMatter/Colophon}
%
%\input{FrontMatter/Thanks}
%
%\input{FrontMatter/Dedication}
%
%\input{FrontMatter/Indexes}
%
%\input{FrontMatter/SommarioAbstract}
%
% ------------------------------------------------------------------------ %
% 	MAINMATTER
% ------------------------------------------------------------------------ %
%
\mainmatter
%
% ------------------------------------------------------------------------ %
% !TEX encoding = UTF-8
% !TEX TS-program = pdflatex
% !TEX root = ../Project.tex
% !TEX spellcheck = en-EN
% ------------------------------------------------------------------------ %
%
% ------------------------------------------------------------------------ %
% 	CHAPTER TITLE
% ------------------------------------------------------------------------ %
%
\chapter{Design and execution\\of the study}
%
%
% ------------------------------------------------------------------------ %
%
\section{Evaluation method adopted}
For the evaluation of the usability of ``The Big Family" site is used the user testing method. With this approach usability properties are retrieved analyzing the interaction between some representatives of real users and the system. The test is performed as the developers of the site want to gather data in order to improve their product.
%
% ------------------------------------------------------------------------ %
%
\section{Task scenarios defined for the test}
\begin{itemize}
\item Task 1: You stumbled upon the website of ``The Big Family", understand what is the scope of this association.
\item Task 2: You are helping a friend looking for services for disabled preople near Certaldo. Identify the ones offered by the association.
\item Task 3: You want to write an email to have some more information about the association. Find the address.
\item Task 4: You need to contact someone working in the Pet Therapy service. Find a telephone number to call.
\item Task 5: You want to visit the association's site nearest to your house. Find where it is.
\end{itemize}
%
% ------------------------------------------------------------------------ %
%
\section{Partecipants for the test}
For this test the partecipants are 5 users and 1 moderators.\\
The users are recruited in order to best match the possible target audience for the system, and this was done considering that the site is about an association that holds some children care center on the territory.\\
The user profiles found in the audience of the association are: 
\begin{itemize}
\item Young adult
\item Adult parent of a child
\item Retired
\end{itemize}
The user goal is to:
\begin{itemize}
\item Find information about the association, since one of their relatives is a children with disability and the association could be useful for him/her
\end{itemize}
The moderator is one the developers of the site, alternately, depending on the disponibility of them.
%
% ------------------------------------------------------------------------ %
%
\section{Usability variables to be measured}
For each user and task, the moderators gather quantitative and qualitative indicators.\\
Quantitative indicators:
\begin{itemize}
\item Efficiency (time used for every task)
\item Effectiveness (task completion (with or without assistance))
\item Number of errors
\item Task success rate
\end{itemize}
Qualitative data:
\begin{itemize}
\item What is liked/disliked
\item Disorientation (information not found)
\item Frustration
\end{itemize}
%
% ------------------------------------------------------------------------ %
%
\section{How the test was performed}
For the execution of the test a typical context of use of the site is simulated (i.e on the dining room table or in the living room) and every user is requested to work with a laptop to perform the tasks predefined by the moderator.\\
Before starting the test, its steps and its purpose are explained, the user is set at his ease and is told that he can leave whenever he wants.
The sheet with the tasks is given to the user that can start to read and to ask what he doesn't understand of it or whatever comes to his mind. At this point the actual test can start. Since every profile of the audience can need and perform any action on the site, all predefined tasks are requested to every user. As long as the user performs the various tasks the moderator is present to gather the data needed and to observe any obstacle for the users to reach their goals.\\
After the test a simple questionnaire is delivered to the user to understand his opinion about the usability of the system.

%
%
% --------------------------------END------------------------------------- %

%
% ------------------------------------------------------------------------ %
% !TEX encoding = UTF-8
% !TEX TS-program = pdflatex
% !TEX root = ../Project.tex
% !TEX spellcheck = en-EN
% ------------------------------------------------------------------------ %
%
% ------------------------------------------------------------------------ %
% 	CHAPTER TITLE
% ------------------------------------------------------------------------ %
%
\chapter{Results}

\section{Test Rules}
For execution of the test some rules about the task performance are used.
Starting from the scoring method considered for the task success, the one used is the following:
\begin{itemize}
\item Complete success (without assistance) = 1
\item Partial success, or if assistance given = 0.5
\item Gives up or wrong answer = 0
\end{itemize}
For the determination of unsuccessful tasks, a task is considered unsuccessful if one of the following condition is matched:
\begin{itemize}
\item The user give up on trying to complete the task
\item Three wrong paths, or three attempts from the start, but the user is free to persevere, even though the task is considered unsuccessful
\item The cut-off time (threshold) is elapsed (for this test 4 minutes are chosen as cut-off time)
\end{itemize}
At last, action is considered an error if:
\begin{itemize}
\item The user enter incorrect data into a form field
\item The user makes the wrong choice in a menu or drop-down list
\item The user takes an incorrect sequence of actions
\item The user fails to take a key action
\end{itemize}
%
% ------------------------------------------------------------------------ %
%
\section{Documentation of the test}
All the data retrieved by the moderators and compiled by the users during the tests are presented below.

As reminder these ones are the tasks asked to the users:
\begin{itemize}
\item Task 1: You stumbled upon the website of “The Big Family", understand what is the scope of this association.
\item Task 2: You are helping a friend looking for services for disabled people near Certaldo. Identify the ones offered by the association.
\item Task 3: You want to write an email to have some more information about the association. Find the address.
\item Task 4: You need to contact someone working in the Pet Therapy service. Find a telephone number to call.
\item Task 5: You want to visit the association’s site nearest to your house. Find where it is.
\end{itemize}

\subsection{Task record sheet}
In the next page the data retrived by moderators during the test, collected in the task record sheet. For every user and every task, the moderators observed:
\begin{itemize}
\item Task time
\item Task completion
\item Number of errors
\item Possible observation on the the behaviour of the user or comments
\end{itemize}

\newpage
\begin{center}
\makebox[\textwidth][c]{\includegraphics[width=1.11\textwidth]{Documents/taskrecord}}
\end{center}

\newpage
\subsection{Post test questionnaires}
These questionnaires are provided to the users after the execution of the test, in order to retrieve the overall feelings perceived by them regarding the tasks (like difficulty, disorientation...). The questionnaires are filled in by each one of the five people involved in the testing and are listed in the same order as in the task record sheet. The questionnaire used is the DEEP (Design oriented evaluation of perceived web usability), that is composed of questions about content, structure and navigation of the site.

\begin{center}
\makebox[\textwidth][c]{\includegraphics[width=1.09\textwidth]{Documents/post1}}
\makebox[\textwidth][c]{\includegraphics[width=1.09\textwidth]{Documents/post2}}
\makebox[\textwidth][c]{\includegraphics[width=1.09\textwidth]{Documents/post3}}
\makebox[\textwidth][c]{\includegraphics[width=1.09\textwidth]{Documents/post4}}
\makebox[\textwidth][c]{\includegraphics[width=1.09\textwidth]{Documents/post5}}
\end{center}









\section{Aggregate data}
Once all the testing is finished and the information is retrieved by the moderators, all the data need to be put together in order to recapture useful information for the improvement of the site.

\subsection{Quantitative data}

\subsubsection{Average time}
\begin{itemize}

\item Task 1: You stumbled upon the website of “The Big Family", understand what is the scope of this association.\\
Average time: \textbf{0 min 47 sec}

\item Task 2: You are helping a friend looking for services for disabled people near Certaldo. Identify the ones offered by the association.\\
Average time: \textbf{0 min 21 sec}

\item Task 3: You want to write an email to have some more information about the association. Find the address.\\
Average time: \textbf{0 min 27 sec}

\item Task 4: You need to contact someone working in the Pet Therapy service. Find a telephone number to call.\\
Average time: \textbf{0 min 17 sec}

\item Task 5: You want to visit the association’s site nearest to your house. Find where it is.\\
Average time: \textbf{0 min 48 sec}

\item Average time considering all tasks: \textbf{0 min 32 sec}
\end{itemize}

\subsubsection{Total number of error}
\begin{itemize}
\item Task 1: \textbf{4}
\item Task 2: \textbf{0}
\item Task 3: \textbf{3}
\item Task 4: \textbf{0}
\item Task 5: \textbf{5}
\item Total: \textbf{12}
\end{itemize}

\subsubsection{Task success rate}
\begin{itemize}
\item Task 1: (4*1/5*100) = \textbf{80\%}
\item Task 2: (5*1/5*100) = \textbf{100\%}
\item Task 3: (5*1/5*100) = \textbf{100\%}
\item Task 4: (5*1/5*100) = \textbf{100\%}
\item Task 5: (4*1/5*100) = \textbf{80\%}
\item General task success rate: (23*1/25*100) = \textbf{92\%}
\end{itemize}

\subsection{Qualitative data}
Regarding qualitative data the information is retrieved from comments and observed behaviour of the user and using the answers to the questionnaires.

\subsubsection{Data retrieved from observation}
From the observation made by the moderators, tasks 2 and 4 didn't gave particular problems to the users, except for an advice to make Links motr highlighted, but it's an advice that could be extended to all site because the whole site is built with the same style.\\
Task 1, 3 and 5 gave more problems to the users:
\begin{itemize}
\item Task 1: Gave problems in understanding what "Who We Are" stands for, and it was confused with "Services".
\item Task 3: Gave problems in finding "Contact Us" link to the page requested by the task, and users wanted to have it more visible in the home page.
\item Task 5: Gave some problems in finding "Locations" and in one case the user chose contacts page in order to know the locations, before using the correct link.
\end{itemize}

\subsubsection{Data retrieved from the questionnaires}
Having all questionnaires filled up is possible to compute a questionnaire that contains the average answer given by the users (if the average answer is between two boxes, the pessimistic one is chosen).

\begin{center}
\makebox[\textwidth][c]{\includegraphics[width=1.09\textwidth]{Documents/postavg}}
\end{center}

The average satisfaction of the users towards the site is good. The only thing to note is the ease with which the users find the information needed, that doesn't have a good evaluation.


\subsection{Old results}
The tasks' results show that everyone was able to complete all the tasks in a reasonable time and with zero errors. The only noteworthy data are three:
\begin{itemize}
\item On the first task one person took a long time (2 minutes and 21 seconds), made three errors and noted that a link to the ``Who We Are'' page could be added to the content of the homepage. On average this task took more time than the other ones to everybody and another person did one error.
\item On the third task one person made two errors and it took 48 seconds to complete it (not incredibly higher than average time). Also was noted that ``Contact Us'' landmark could be more noticeable.
\item On the fifth task one person made four errors and it took 2 minutes and 11 seconds to complete (pretty higher than average). It was also suggested that the ``Locations'' landmark could be more noticeable.
\end{itemize}
The data collected by survey confirms the task's result, with a good overall impression of the structure and information architecture, content and navigation. Only one person replied neutral at the statement ``It was easy to find the information I needed''.




% -----------------------------END------------------------------------- %
%
% ------------------------------------------------------------------------ %
% !TEX encoding = UTF-8
% !TEX TS-program = pdflatex
% !TEX root = ../Project.tex
% !TEX spellcheck = en-EN
% ------------------------------------------------------------------------ %
%
% ------------------------------------------------------------------------ %
% 	CHAPTER TITLE
% ------------------------------------------------------------------------ %
%
\chapter{Discussion of results}
This chapter is the recap of the results of the testing and a list of recommended interventions to the site, for its improvement.

\section{Problem report}

\subsection{Quantitative data}
Concerning quantitative data retrieved from the test. The site appear intuitive and user friendly since all tasks are performed in less than one minute, compared to the 4 minutes of the cut-off time (never reached).\\
The test registered 12 error from the users, all contained in the execution of three tasks (first, third and fifth). 4 errors in the the first task mean that ``Who We Are'' page title is not effectively understandable and doesn't explain well the content of the page. 3 errors in the third task are related to the difficulty in finding the ``Contact Us'' landmark since it's in the footer at the end of the page. 5 errors in the last task could be defined by the fact that ``Contact Us'' page contains a map, so user that already had done the task about contacts of the association, could be led to use that page instead of ``Locations''.\\
The task success rate is high even in the tasks that gave more problems (minimum task success rate is 80\%), it means that the site is simple and well structured, event tough some initial disorientation or misunderstandings by the users can occur.

\subsection{Qualitative data}
Using the qualitative data obtained by the comments, observations and questionnaires other results can be computed.\\
From the comments of the users and moderator's observations during the execution of the test, the main problem that came up is with the tracing of the right link for the completion of the task.
From the questionnaires emerged the same problem as in the comments and observations, but some users evaluated coldly also the relevance and utility of the text and the capacity of the site to help the users to find what they were looking for.

\section{Final recommendations}
The final recommendations of these report will be divided in three categories:
\begin{itemize}
\item Priority 1: Mandatory and urgent interventions
\item Priority 2: Needed interventions but not urgent
\item Priority 3: Hopeful interventions
\end{itemize}

\subsection{Priority 1}
With higher priority is recommended to change the position of the ``Contact Us'' landmark to a place in the site that is immediately visible for the users. Another urgent intervention to do is to change the name of the page ``Who We Are'' in something like ``What Is The Association'' or ``History And Values'' to highlight the fact that the page describes the history and the scope of the association.

\subsection{Priority 2}
With priority 2 it is recommended to move the map that is located in ``Contact Us'' into ``Locations'' and to add a link from the page of contacts to the map.

\subsection{Priority 3}
With the less important priority, the intervention recommended is to change the style, the color or thedecoration of the links, in order to make them more visible. 
%
% -----------------------------END------------------------------------- %

%
%\appendix
%
%\input{MainMatter/Appendix}
%
% ------------------------------------------------------------------------ %
% 	BACKMATTER
% ------------------------------------------------------------------------ %
%
%\backmatter
%
%\input{BackMatter/Bibliography}
%
% ------------------------------------------------------------------------ %
% 	END DOCUMENT
% ------------------------------------------------------------------------ %
%
\end{document}
%
% ------------------------------------------------------------------------ %